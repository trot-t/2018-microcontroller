\documentclass[a4paper,11pt]{article}
\usepackage[T2A]{fontenc}
\usepackage[utf8]{inputenc}
\usepackage[english, russian]{babel}
\usepackage{tikz}
\usepackage[european,cuteinductors,smartlabels]{circuitikz} 

% Конец преамбулы
\begin{document}
%\section{Работа со временами в микроконтроллере} 
\section{Таймеры, теневые регистры, прерывания, основной поток вычислений} 
\begin{tikzpicture}[scale=1.1]
\newcommand{\D}{2.1}
\newcommand{\DD}{2.5}
\newcommand{\baset}{2}  % основной уровень таймера
\newcommand{\basem}{1.2} % основной уровень подпрограммы
\newcommand{\lineL}{0.8} % длительность подпрограммы
\newcommand{\stub}{0.15} % высота/ширина пенька прерывания
\draw[->,>=stealth'] (0,0) -- (11,0) node[below] {$t$};
\draw[red,very thick] ({0},{\baset}) -- ({0+\D},{\baset+\D}) -- ({0+\D},{\baset}) -- ({0+2*\D},{\baset+ \D}) -- ({0+2*\D},{\baset}) -- ({0+2*\D+\DD},{\baset+\DD}) --
({0+2*\D+\DD},{\baset}) -- ({0+2*\D+2*\DD},{\baset+\DD}) -- ({0+2*\D+2*\DD},{\baset}) -- ({0+2*\D+2.5*\DD},{\baset+0.5*\DD});
	\draw[blue,thick,->] (0,0.6) -- (8,0.6) node[right] {основная программа};
	\draw[green] (7.6,{\basem}) node[right] {обработка прерывания};
	\draw[red]  (10,{\baset}) node[right] {таймер};
	\draw ({0+\D},0) -- ({0+\D},{\stub}) -- ({0+\D+\stub},{\stub}) --  ({0+\D+\stub},0);
	\draw ({0+2*\D},0) -- ({0+2*\D},{\stub}) -- ({0+2*\D+\stub},{\stub}) --  ({0+2*\D+\stub},0);
	\draw ({0+2*\D+\DD},0) -- ({0+2*\D+\DD},{\stub}) -- ({0+2*\D+\DD+\stub},{\stub}) --  ({0+2*\D+\DD+\stub},0);
	\draw ({0+2*\D+2*\DD},0) -- ({0+2*\D+2*\DD},{\stub}) -- ({0+2*\D+2*\DD+\stub},{\stub}) --  ({0+2*\D+2*\DD+\stub},0);

	\draw[green] ({0+\D + \lineL},{\basem}) to[L,color=green] ({0+\D},{\basem});
\draw[very thin,->,>=stealth'] ({\D},{\baset}) -- ({\D},{\basem});
	\draw[thin,yellow] ({0},{\D+\D/4}) -- ({2*\D},{\D+\D/4});
	\draw[thin,green] ({0},{\D+\D/3}) -- ({2*\D},{\D+\D/3});
	\draw[thin,red] ({0},{\D+3*\D/4}) -- ({2*\D},{\D+3*\D/4});
\draw[very thin,->,>=stealth'] ({\D+\lineL},{\basem}) -- ({\D+\lineL},{\D+3*\DD/4});
\draw[very thin,->,>=stealth'] ({\D+\lineL+0.03},{\basem}) -- ({\D+\lineL+0.03},{\D+\DD/3});
\draw[very thin,->,>=stealth'] ({\D+\lineL+0.06},{\basem}) -- ({\D+\lineL+0.06},{\D+\DD/6});
        \draw[thin,yellow] ({2*\D},{\D+\DD/6}) -- ({2*\D+\DD},{\D+\DD/6}); 
	\draw[thin,green] ({2*\D},{\D+\DD/3}) -- ({2*\D+\DD},{\D+\DD/3});
	\draw[thin,red] ({2*\D},{\D+3*\DD/4}) -- ({2*\D+\DD},{\D+3*\DD/4});
\draw[green] ({2*\D+\lineL},{\basem}) to[L,color=green] ({2*\D},{\basem});
\draw[very thin,->,>=stealth'] ({2*\D},{\baset}) -- ({2*\D},{\basem});
\draw[very thin,->,>=stealth']  ({2*\D + \lineL},{\basem}) -- ({2*\D + \lineL},{\D+2*\DD/3});
\draw[very thin,->,>=stealth']  ({2*\D + \lineL+0.03},{\basem}) -- ({2*\D + \lineL+0.03},{\D+\DD/2});
\draw[very thin,->,>=stealth']  ({2*\D + \lineL+0.06},{\basem}) -- ({2*\D + \lineL+0.06},{\D+\DD/6*1.3});
\draw[yellow, dashed]  ({\D + \lineL}, {\D+\DD/6}) -- ({2*\D},{\D+\DD/6});
\draw[green, dashed]  ({\D + \lineL}, {\D+\DD/3}) -- ({2*\D},{\D+\DD/3});
\draw[red, dashed]  ({\D + \lineL}, {\D+3*\DD/4}) -- ({2*\D},{\D+3*\DD/4});
	\draw[thin,yellow] ({2*\D+\DD},{\D+\DD/6*1.3}) -- ({2*\D+2*\DD},{\D+\DD/6*1.3});
	\draw[thin,green] ({2*\D+\DD},{\D+\DD/2}) -- ({2*\D+2*\DD},{\D+\DD/2});
        \draw[thin,red] ({2*\D+\DD},{\D+2*\DD/3}) -- ({2*\D+2*\DD},{\D+2*\DD/3});
\draw[yellow, dashed]  ({2*\D + \lineL}, {\D+\DD/6*1.3}) -- ({2*\D+\DD},{\D+\DD/6*1.3});
\draw[green, dashed]  ({2*\D + \lineL}, {\D+\DD/2}) -- ({2*\D+\DD},{\D+\DD/2});
\draw[red, dashed]  ({2*\D + \lineL}, {\D+2*\DD/3}) -- ({2*\D+\DD},{\D+2*\DD/3});
\draw[green] ({2*\D+\DD+\lineL},{\basem}) to[L,color=green] ({2*\D+\DD},{\basem});
\draw[very thin,->,>=stealth'] ({2*\D+\DD},{\baset}) -- ({2*\D+\DD},{\basem});
\draw[very thin,->,>=stealth']  ({2*\D+\DD + \lineL},{\basem}) -- ({2*\D+\DD + \lineL},{\D+2*\DD/3}); 
\end{tikzpicture}
\vspace{0.5cm}
$$
\textcyrillic{делитель частоты} = \frac{\textcyrillic{исходная частота}}{\textcyrillic{коэффициент деления}} - 1
$$

Делитель частоты физически реализуется счетчиком пропусков событий. 0 пропусков соответствуют делению на 1. Один пропуск соответствуют делению на 2.  

\end{document}
